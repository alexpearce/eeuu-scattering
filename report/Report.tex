% list options between brackets
\documentclass[]{article}
% list packages between braces
\usepackage{atlasphysics, amsmath}

% user-defined commands here

\begin{document}

\title{The $\ee \rarrow \mumu$ Cross Section in the Standard Model}
\author{Alex Pearce}
\date{\today}
\maketitle

\begin{abstract}
	The Standard Model's (SM) prediction of particles beyond those initially considered by quantum electrodynamics (QED) has yielded excellent results. The Super Proton Synchrotron (SPS) at CERN recently detected both the $\Wboson$ bosons and the $\Zzero$ boson via the $\antibar{p}$ mechanism (Rubbia, van der Meer et al.). We performed a numerical integration of the differential cross section of the $\ee \rarrow \mumu$ scattering process which may produce $\Zzero$ bosons in the hope that the proposed Large Electron-Positron collider (LEP) will verify this channel of $\Zzero$ production. A distinct $\Zzero$ resonance around the $\Zzero$ mass of 91.8\GeV was found with a cross section $\sigma=9.4\inb$.
\end{abstract}

%\tableofcontents

\section{Introduction}
The proposition of three mediators of the weak nuclear force, the $\Wplus$, $\Wminus$ and $\Zzero$ bosons, has been all but proven by the current team at CERN operating the SPS. The suggestion of $\Zzero$ production via electron-positron pairs is now becoming of interest to experimentalists. The process itself manifests itself by an electron-position pair ($e^{-}e^{+}$) annihilating, forming either a virtual photon or $\Zzero$ boson, then a muon-antimuon pair ($\mu^{-}\mu^{+}$) being produced.

\subsection{History}
History.

\section{\LaTeX}
\TeX

\subsection{Usage}
Usage.

\begin{thebibliography}{9}
  % type bibliography here
\end{thebibliography}

\end{document}