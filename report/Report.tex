\documentclass[]{article}
% ATLAS macros
\usepackage{atlasphysics}
% Nice maths macros
\usepackage{amsmath}
% Feynman diagrams
\usepackage{feynmp}
% Figures and floats
\usepackage{graphicx,subfig}

% Read .1 file extension as .mps.
\DeclareGraphicsRule{.1}{mps}{*}{}

% user-defined commands here

\begin{document}

\title{The $\ee \rarrow \mumu$ Cross Section in the Standard Model}
\author{Alex Pearce}
\date{\today}
\maketitle


\begin{abstract}
The Standard Model's (SM) prediction of particles beyond those initially considered by quantum electrodynamics (QED) has yielded excellent results. The Super Proton Synchrotron (SPS) at CERN recently detected both the $\Wboson$ bosons and the $\Zzero$ boson via the $\antibar{p}$ mechanism (Rubbia, van der Meer et al.). We performed a numerical integration of the differential cross section of the $\ee \rarrow \mumu$ scattering process, which may produce $\Zzero$ bosons, in the hope that the proposed Large Electron-Positron collider (LEP) will verify this channel of $\Zzero$ production. A distinct $\Zzero$ resonance around the $\Zzero$ mass of 91.8\GeV was found with a cross section $\sigma=9.4\inb$.
\end{abstract}


%\tableofcontents


\section{Introduction}\label{sec:intro}

The proposition of three mediators of the weak nuclear force, the $\Wplus$, $\Wminus$ and $\Zzero$ bosons, has been all but proven by the current team at CERN operating the SPS. The suggestion of $\Zzero$ production via electron-positron pairs is now becoming of interest to experimentalists. The process is manifested by an electron-position pair ($e^{-}e^{+}$) annihilating, forming either a virtual photon or $\Zzero$ boson, and then a muon-antimuon pair ($\mu^{-}\mu^{+}$) being produced.

This interaction is described by the Feynman diagram in figure \ref{fig:feynsgammaz}. The scattering is also described by a t-channel diagram (in figure \ref{fig:feyntgammaz}), however we proceed by analysing only the s-channel as it is only this channel via which we may measure resonances and new unstable particles. Note that a u-channel diagram also exists, but as it is simply a swapping of the outgoing particles' momenta in the t-channel, we ignore this also.

The use of Feynman diagrams is that we may apply the Feynman rules to them to produce a  matrix element $\mathcal{M}$. This in fact corresponds to a differential cross section $\frac{\d{\sigma}}{\d{\Omega}}$ which may be integrated to find the total cross section $\sigma$, which is measurable by a detector.

The methods of integration used are explored in the next section, along with a study of the differential cross sections in order to judge the effectiveness of numerical integration upon them. The results are presented and analysed in section \ref{sec:results}, then a discussion of the kinematic variables follows in section \ref{sec:variables}. Finally, TODO is considered in section \ref{sec:extension}


\section{Integration of the Differential $\frac{\d{\sigma}}{\d{\Omega}}$}\label{sec:integration}

History.


\section{Results and Analysis}\label{sec:results}

\TeX


\section{Kinematic Variables}\label{sec:variables}

Discussion of $\cos{\theta}$ and $p_{T} = \lvert\vec{p}_{f}\rvert\sin{\theta}$.


\section{Extension}\label{sec:extension}

TODO Extension of the problem.


\section{Figures}

\begin{figure}[h]
	\vspace{10pt}
	% Diagram unit length.
	\unitlength = 1mm
	\centering
	\subfloat[s-channel]{
		\label{fig:feynsgammaz}
		\begin{fmffile}{sgammazcrossing}
		  \begin{fmfgraph*}(40,25)
		    \fmfleft{i1,i2}
		    \fmfright{o1,o2}
		    \fmflabel{$e^-$}{i1}
		    \fmflabel{$e^+$}{i2}
		    \fmflabel{$\mu^+$}{o1}
		    \fmflabel{$\mu^-$}{o2}
		    \fmf{fermion}{i1,v1,i2}
		    \fmf{fermion}{o1,v2,o2}
		    \fmf{photon,label=$\gamma^{*}/\Zzero^{(*)}$}{v1,v2}
		  \end{fmfgraph*}
		\end{fmffile}
	}
	\qquad
	\subfloat[t-channel]{
		\label{fig:feyntgammaz}
		\begin{fmffile}{tgammazcrossing}
		  \begin{fmfgraph*}(40,25)
		    \fmfleft{i1,i2}
		    \fmfright{o1,o2}
		    \fmflabel{$e^-$}{i1}
		    \fmflabel{$e^+$}{i2}
		    \fmflabel{$\mu^-$}{o1}
		    \fmflabel{$\mu^+$}{o2}
		    \fmf{fermion}{i1,v1,o1}
		    \fmf{fermion}{o2,v2,i2}
		    \fmf{photon,label=$\gamma^{*}/\Zzero^{(*)}$}{v1,v2}
		  \end{fmfgraph*}
		\end{fmffile}
	}
	\caption{$e^{-}e^{+}\rarrow\mu^{-}\mu^{+}$ scattering via two different channels.}
\end{figure}

\begin{thebibliography}{9}
  % type bibliography here
\end{thebibliography}

\end{document}