\documentclass[]{report}
\usepackage{atlasphysics,amsmath,feynmp, graphicx, subfig}
% Read .1 file extension as .mps.
\DeclareGraphicsRule{.1}{mps}{*}{}
% Diagram unit length.
\unitlength = 1mm

\begin{document}

\tableofcontents

\chapter{Derivations}

\section{Theoretical Cross Section $\sigma$}

\subsection{$\gamma-\gamma$}

Define

$$
\alpha = \frac{g_{e}^{4}}{(8\pi)^{2}s} \sqrt{1-4\varepsilon^{2}}.
$$

The differential cross section then becomes

$$
\frac{\d{\sigma}}{\d{\Omega}} = \alpha (1 + \cos^{2}{\vartheta} + 4\varepsilon^{2}\sin^{2}{\vartheta}).
$$

Making the canonical substitution $\sin^{2}{\vartheta} = 1 - \cos^{2}{\vartheta}$ and integrating over the solid angle $\d{\Omega} = \d{\phi}\d{x}$ (with implicit change of variable $\cos{\vartheta}\ra x$):

\begin{align*}
\sigma &= \alpha \int\limits_{0}^{2\pi}\d{\phi} \int\limits_{-1}^{1} 1 + x^{2} + 4\varepsilon^{2}(1 - x^{2}).
\end{align*}

This is trivially evaluated to

$$
\sigma = \frac{16\pi\alpha}{3}(1 + 2\varepsilon^{2})
$$

\subsection{$\Zzero-\Zzero$}

Define

\begin{align*}
\alpha &= \frac{g_{\Zzero}^{4}}{(32\pi)^{2}s} \frac{\sqrt{1-4\varepsilon^{2}}}{(1-\lambda^{2})^{2} + (\frac{\lambda\Gamma_{\Zzero}}{\rts})},
\\
\beta &= ((C_{V}^{e})^{2} + (C_{A}^{e})^{2})((C_{V}^{\mu})^{2}),
\\
\Gamma &= ((C_{V}^{e})^{2} + (C_{A}^{e})^{2})((C_{A}^{\mu})^{2})(1-4\varepsilon^{2}),
\\
\Delta &= 8C_{V}^{e}C_{A}^{e}C_{V}^{\mu}C_{A}^{\mu}\sqrt{1-4\varepsilon^{2}}.
\end	{align*}

These definitions produce the differential cross section:

$$
\frac{\d{\sigma}}{\d{\Omega}}^{\Zzero-\Zzero}
  = \alpha(\beta(1+\cos^{2}{\vartheta}+4\varepsilon^{2}\sin^{2}{\vartheta})
    + \Delta(1+\cos^{2}{\vartheta})
    + \Gamma\cos{\vartheta}
  )
$$

Under the substitution $\sin^{2}{\vartheta} \ra 1-\cos^{2}{\vartheta}$ we see that the integral over the solid angle $\d{\Omega} = \d{\phi}\d{x}$ (with implicit change of variable $\cos{\vartheta}\ra x$) becomes

\begin{align*}
\sigma &= \alpha\int\limits_{0}^{2\pi}\d{\phi}\int\limits_{-1}^{1}
	\beta + \beta x^{2} + 4\varepsilon^{2}\beta(1-x^{2}) + \Gamma + \Gamma x^{2} + \Delta x
	\d{x}
  \\
  &= 2\pi\alpha\int\limits_{-1}^{1}
		(1 + 4\varepsilon^{2})\beta + \Gamma + ((1 - 4\varepsilon^{2})\beta + \Gamma)x^{2}
    \d{x}.
\end{align*}

The $\Gamma$ term disappears as it is antisymmetric in the integration limits. After simplification, the integral is evaluated to

$$
\sigma^{\Zzero-\Zzero} = \frac{16\pi\alpha}{3}(\Gamma + (1 + 2\varepsilon^{2})\beta).
$$

\subsection{$\gamma-\Zzero$}

Define

\begin{align*}
\alpha &= \frac{2g_{e}^{2}g_{\Zzero}^{2}}{(16\pi)^{2}s} \frac{(1-\lambda^{2})\sqrt{1-4\varepsilon^{2}}}{(1-\lambda^{2})^{2} + (\frac{\lambda\Gamma_{\Zzero}}{\rts})},
\\
\beta &= C^{e}_{V}C^{\mu}_{V},
\\
\Delta &= 2C^{e}_{A}C^{\mu}_{A}\sqrt{1-4\varepsilon^{2}},
\end	{align*}

These definitions lead to the differential cross section

$$
\frac{\d{\sigma}}{\d{\Omega}}^{\gamma-\Zzero}
  = \alpha(\beta(1+\cos^{2}{\vartheta}+4\varepsilon^{2}\sin^{2}{\vartheta})
    + \Delta\cos{\vartheta}
  )
$$

Under the substitution $\sin^{2}{\vartheta} \ra 1-\cos^{2}{\vartheta}$, we see that the integral over the solid angle $\d{\Omega} = \d{\phi}\d{x}$ (with implicit change of variable $\cos{\vartheta}\ra x$) becomes

\begin{align*}
\sigma &= \alpha \int\limits_{0}^{2\pi}\d{\phi}
	\int\limits_{-1}^{1}
		\beta(1+x^{2}+4\varepsilon^{2}(1- x^{2}))
    	+ \Delta x
	\d{x}
\\
&= 2\pi\alpha \int\limits_{-1}^{1} \beta(1+4\varepsilon^{2})+\beta(1-4\varepsilon^{2})x^{2} + \Delta x \d{x}.
\end{align*}

The antisymmetric $\Delta x$ terms vanishes under integration, giving (after rearranging)s

$$
\sigma^{\gamma-\Zzero} = \frac{16\pi\alpha\beta}{3}(1 + 2\varepsilon^{2}).
$$

\chapter{Misc.}

\section{Feynman Diagrams}

Feynman diagrams for the s-channel scatterings.

\begin{figure}[h]
	\vspace{10pt}
	\centering
	\subfloat[$\gamma-\gamma$]{
		\begin{fmffile}{sgammacrossing}
		  \begin{fmfgraph*}(40,25)
		    \fmfleft{i1,i2}
		    \fmfright{o1,o2}
		    \fmflabel{$e^-$}{i1}
		    \fmflabel{$e^+$}{i2}
		    \fmflabel{$\mu^+$}{o1}
		    \fmflabel{$\mu^-$}{o2}
		    \fmf{fermion}{i1,v1,i2}
		    \fmf{fermion}{o1,v2,o2}
		    \fmf{photon,label=$\gamma$}{v1,v2}
		  \end{fmfgraph*}
		\end{fmffile}
	}
	\qquad
	\subfloat[$\Zzero-\Zzero$]{
		\begin{fmffile}{szcrossing}
		  \begin{fmfgraph*}(40,25)
		    \fmfleft{i1,i2}
		    \fmfright{o1,o2}
		    \fmflabel{$e^-$}{i1}
		    \fmflabel{$e^+$}{i2}
		    \fmflabel{$\mu^+$}{o1}
		    \fmflabel{$\mu^-$}{o2}
		    \fmf{fermion}{i1,v1,i2}
		    \fmf{fermion}{o1,v2,o2}
		    \fmf{boson,label=$\Zzero^{(*)}$}{v1,v2}
		  \end{fmfgraph*}
		\end{fmffile}
	}
	\caption{s-channel $e^{-}e^{+}\rarrow\mu^{-}\mu^{+}$ scattering via different bosons.}
\end{figure}

Feynman diagrams for the t-channel scatterings.

\begin{figure}[h]
	\vspace{10pt}
	\centering
	\subfloat[$\gamma-\gamma$]{
		\begin{fmffile}{tgammacrossing}
		  \begin{fmfgraph*}(40,25)
		    \fmfleft{i1,i2}
		    \fmfright{o1,o2}
		    \fmflabel{$e^-$}{i1}
		    \fmflabel{$e^+$}{i2}
		    \fmflabel{$\mu^-$}{o1}
		    \fmflabel{$\mu^+$}{o2}
		    \fmf{fermion}{i1,v1,o1}
		    \fmf{fermion}{o2,v2,i2}
		    \fmf{photon,label=$\gamma$}{v1,v2}
		  \end{fmfgraph*}
		\end{fmffile}
	}
	\qquad
	\subfloat[$\Zzero-\Zzero$]{
		\begin{fmffile}{tzcrossing}
		  \begin{fmfgraph*}(40,25)
		    \fmfleft{i1,i2}
		    \fmfright{o1,o2}
		    \fmflabel{$e^-$}{i1}
		    \fmflabel{$e^+$}{i2}
		    \fmflabel{$\mu^-$}{o1}
		    \fmflabel{$\mu^+$}{o2}
		    \fmf{fermion}{i1,v1,o1}
		    \fmf{fermion}{o2,v2,i2}
		    \fmf{boson,label=$\Zzero^{(*)}$}{v1,v2}
		  \end{fmfgraph*}
		\end{fmffile}
	}
	\caption{s-channel $e^{-}e^{+}\rarrow\mu^{-}\mu^{+}$ scattering via different bosons.}
\end{figure}

\end{document}